\documentclass[a4paper,10pt]{paper}
\usepackage[utf8x]{inputenc}
\usepackage{eurosym}
\usepackage{listings}
\usepackage{hyperref}
 \usepackage{wasysym}
\usepackage{graphicx}
\usepackage{rotating}
\usepackage{float}
\usepackage{lscape}

%%%%%%%%%%%%%%%%%%%%%%%%%%%  for headers   %%%%%%%%%%%%%%%%%%%%%%%%%%%%
\usepackage{fancyhdr}
\pagestyle{fancy} 
\fancyhead{} % Clear all header fields 
\fancyhead[L]{\includegraphics[height=20pt]{uniStuttgartLogo}}% 
\fancyhead[R]{\bfseries{Institut für Technische Verbrennung}}
\fancyfoot{} % Clear all footer fields 
%%%%%%%%%%%%%%%%%%%%%%%%%%%%%%%%%%%%%%%%%%%%%%%%%%%%%%%%%%%%%%%%%%%%%%%

\title{12.09.2012-14.09.2012}
\author{Gizem Inci}

\begin{document}

\begin{center}
        \section*{\textbf {GIZEM INCI\\
                Molecular Dynamics - Monte Carlo}}
\end{center}

 \section*{Molecular Dynamics Simulation Method}
Molecular dynamics (MD) simulation is technique by which one generates the atomic trajectories of a system of N particles
by numerical integration of Newton’s equation of motion, for a specificinteratomic potential, with certain initial condition 
(IC) and boundary condition (BC).

 \section*{Monte Carlo Simulation Method}
The Monte Carlo (MC) simulation method is a stochastic strategy that relies on probabilities. MC is based on the idea of calculating
the averages of statistical mechanics with the use of numerical integration.

 \section*{MD vs MC}

 \subsection*{Similarities}

\begin{enumerate}
	\item Utilization of classicial force fields for the potential energy terms.
	\item The force field controls the total energy (MC) and forces (MD), which determine the evolution of the system.
	\item Implementation of periodic boundary condition.
\end{enumerate}

 \subsection*{Differences}

\begin{enumerate}
	\item Sampling the configuration space.
	\begin{enumerate}
		\item In MC simulation method, new configuration is generated by selecting a random molecule, translating it, rotating it, and performing any structural variation.
		\item In MD simulation method, new configurations are generated by application of Newton's equations of motion to all atoms simulataneously over a small time step
to determine the new atomic positions and velocities.
	\end{enumerate}
	\item MD evaluates different configurational properties and dynamic quantities which cannot generally be obtained by MC. 
	\item In MC simulation method, there is a large probability of selecting random moves for which two or more molecules overlap (especially for rotations near the center of molecules with long tails),
leading to large number of rejected moves and a decrease in efficiency of sampling. 
	\item MD handles collective motions better than MC.
	\item Determination of transport properties such as viscosity coefficient,, is largely onyl possible using MD, as MC lacks an objective definition of time.
	\begin{enumerate}
		\item MC does not offer time evolution of the system in a form of suitable for viewing.
		\item MD simulates the time evolution of the molecular system and provides one with the actual trajectory of the system.
	\end{enumerate}
\end{enumerate}

\end{document}
